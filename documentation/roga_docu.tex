%%% File-Information {{{
%%% Filename: roga_docu.tex
%%% Purpose: project rteport
%%% Time-stamp: <2004-06-30 18:19:32 mp>
%%% Authors: The LaTeX@TUG-Team [http://latex.tugraz.at/]:
%%%          Karl Voit (vk), Michael Prokop (mp), Stefan Sollerer (ss)
%%% History:
%%%   20050914 (ss) correction of "backref=true" to "backref" due to hyperref documentation
%%%   20040630 (mp) added comments to foldmethod at end of file
%%%   20040625 (vk,ss) initial version
%%%
%%% Notes:
%%%
%%%
%%%
%%% }}}
%%%%%%%%%%%%%%%%%%%%%%%%%%%%%%%%%%%%%%%%%%%%%%%%%%%%%%%%%%%%%%%%%%%%%%%%%%%%%%%%
%%% main document {{{

\documentclass[
a4paper,     %% defines the paper size: a4paper (default), a5paper, letterpaper, ...
% landscape,   %% sets the orientation to landscape
% twoside,     %% changes to a two-page-layout (alternatively: oneside)
% twocolumn,   %% changes to a two-column-layout
% headsepline, %% add a horizontal line below the column title
% footsepline, %% add a horizontal line above the page footer
% titlepage,   %% only the titlepage (using titlepage-environment) appears on the first page (alternatively: notitlepage)
% parskip,     %% insert an empty line between two paragraphs (alternatively: halfparskip, ...)
% leqno,       %% equation numbers left (instead of right)
% fleqn,       %% equation left-justified (instead of centered)
% tablecaptionabove, %% captions of tables are above the tables (alternatively: tablecaptionbelow)
% draft,       %% produce only a draft version (mark lines that need manual edition and don't show graphics)
% 10pt         %% set default font size to 10 point
% 11pt         %% set default font size to 11 point
12pt         %% set default font size to 12 point
]{scrartcl}  %% article, see KOMA documentation (scrguide.dvi)



%%%%%%%%%%%%%%%%%%%%%%%%%%%%%%%%%%%%%%%%%%%%%%%%%%%%%%%%%%%%%%%%%%%%%%%%%%%%%%%%
%%%
%%% packages
%%%

%%%
%%% encoding and language set
%%%d

%%% ngerman: language set to new-german
\usepackage{ngerman}

%%% babel: language set (can cause some conflicts with package ngerman)
%%%        use it only for multi-language documents or non-german ones
% \usepackage[ngerman]{babel}

%%% inputenc: coding of german special characters
\usepackage[utf8]{inputenc}

%%% fontenc, ae, aecompl: coding of characters in PDF documents
\usepackage[T1]{fontenc}
\usepackage{ae,aecompl}

%%%
%%% technical packages
%%%

%%% amsmath, amssymb, amstext: support for mathematics
%\usepackage{amsmath,amssymb,amstext}

%%% psfrag: replace PostScript fonts
\usepackage{psfrag}

%%% listings: include programming code
%\usepackage{listings}

%%% units: technical units
%\usepackage{units}

%%%
%%% layout
%%%

%%% scrpage2: KOMA heading and footer
%%% Note: if you don't use this package, please remove
%%%       \pagestyle{scrheadings} and corresponding settings
%%%       below too.
\usepackage[automark]{scrlayer-scrpage}


%%%
%%% PDF
%%%

\usepackage{ifpdf}

%%% Should be LAST usepackage-call!
%%% For docu on that, see reference on package ``hyperref''
\ifpdf%   (definitions for using pdflatex instead of latex)

  %%% graphicx: support for graphics
  \usepackage[pdftex]{graphicx}

  \pdfcompresslevel=9

  %%% hyperref (hyperlinks in PDF): for more options or more detailed
  %%%          explanations, see the documentation of the hyperref-package
  \usepackage[%
    %%% general options
    pdftex=true,      %% sets up hyperref for use with the pdftex program
    %plainpages=false, %% set it to false, if pdflatex complains: ``destination with same identifier already exists''
    %
    %%% extension options
    backref,      %% adds a backlink text to the end of each item in the bibliography
    pagebackref=false, %% if true, creates backward references as a list of page numbers in the bibliography
    colorlinks=true,   %% turn on colored links (true is better for on-screen reading, false is better for printout versions)
    %
    %%% PDF-specific display options
    bookmarks=true,          %% if true, generate PDF bookmarks (requires two passes of pdflatex)
    bookmarksopen=false,     %% if true, show all PDF bookmarks expanded
    bookmarksnumbered=false, %% if true, add the section numbers to the bookmarks
    %pdfstartpage={1},        %% determines, on which page the PDF file is opened
    pdfpagemode=None         %% None, UseOutlines (=show bookmarks), UseThumbs (show thumbnails), FullScreen
  ]{hyperref}


  %%% provide all graphics (also) in this format, so you don't have
  %%% to add the file extensions to the \includegraphics-command
  %%% and/or you don't have to distinguish between generating
  %%% dvi/ps (through latex) and pdf (through pdflatex)
  \DeclareGraphicsExtensions{.pdf}

\else %else   (definitions for using latex instead of pdflatex)

  \usepackage[dvips]{graphicx}

  \DeclareGraphicsExtensions{.eps}

  \usepackage[%
    dvips,           %% sets up hyperref for use with the dvips driver
    colorlinks=false %% better for printout version; almost every hyperref-extension is eliminated by using dvips
  ]{hyperref}

\fi


%%% sets the PDF-Information options
%%% (see fields in Acrobat Reader: ``File -> Document properties -> Summary'')
%%% Note: this method is better than as options of the hyperref-package (options are expanded correctly)
\hypersetup{
  pdftitle={}, %%
  pdfauthor={}, %%
  pdfsubject={}, %%
  pdfcreator={Accomplished with LaTeX2e and pdfLaTeX with hyperref-package.}, %%
  pdfproducer={}, %%
  pdfkeywords={} %%
}


%%%%%%%%%%%%%%%%%%%%%%%%%%%%%%%%%%%%%%%%%%%%%%%%%%%%%%%%%%%%%%%%%%%%%%%%%%%%%%%%
%%%
%%% user defined commands
%%%

%%% \mygraphics{}{}{}
%% usage:   \mygraphics{width}{filename_without_extension}{caption}
%% example: \mygraphics{0.7\textwidth}{rolling_grandma}{This is my grandmother on inlinescates}
%% requires: package graphicx
%% provides: including centered pictures/graphics with a boldfaced caption below
%%
\newcommand{\mygraphics}[3]{
\begin{figure}[!h]
  \begin{center}
    \includegraphics[width=#1, keepaspectratio=true]{#2} \\
    \caption{#3}\label{fig:#2}
  \end{center}
\end{figure}

}

%%%%%%%%%%%%%%%%%%%%%%%%%%%%%%%%%%%%%%%%%%%%%%%%%%%%%%%%%%%%%%%%%%%%%%%%%%%%%%%%
%%%
%%% define the titlepage
%%%

% \subject{}   %% subject which appears above titlehead
% \titlehead{} %% special heading for the titlepage

%%% title
\title{CAT-MOUSE-CHEESE}

%%% author(s)
\author{Torben Miller (3164082)\ \ \ \ \ \and
Fabian Biedlingmaier (3224303)\ \ \ \ \ \ \and
Nico Hartlieb (3155952)}

%%% date
\date{Heidelberg, am \today{}}

% \publishers{}

% \thanks{} %% use it instead of footnotes (only on titlepage)

% \dedication{} %% generates a dedication-page after titlepage


%%% uncomment following lines, if you want to:
%%% reuse the maketitle-entries for hyperref-setup
%\newcommand\org@maketitle{}
%\let\org@maketitle\maketitle
%\def\maketitle{%
%  \hypersetup{
%    pdftitle={\@title},
%    pdfauthor={\@author}
%    pdfsubject={\@subject}
%  }%
%  \org@maketitle
%}


%%%%%%%%%%%%%%%%%%%%%%%%%%%%%%%%%%%%%%%%%%%%%%%%%%%%%%%%%%%%%%%%%%%%%%%%%%%%%%%%
%%%
%%% set heading and footer
%%%

%%% scrheadings default:
%%%      footer - middle: page number
\pagestyle{scrheadings}

%%% user specific
%%% usage:
%%% \position[heading/footer for the titlepage]{heading/footer for the rest of the document}

%%% heading - left
% \ihead[]{}

%%% heading - center
% \chead[]{}

%%% heading - right
% \ohead[]{}

%%% footer - left
% \ifoot[]{}

%%% footer - center
% \cfoot[]{}

%%% footer - right
% \ofoot[]{}



%%%%%%%%%%%%%%%%%%%%%%%%%%%%%%%%%%%%%%%%%%%%%%%%%%%%%%%%%%%%%%%%%%%%%%%%%%%%%%%%
%%%
%%% begin document
%%%

\begin{document}

 \pagenumbering{roman} %% small roman page numbers

%%% include the title
% \thispagestyle{empty}  %% no header/footer (only) on this page
 \maketitle

%%% start a new page and display the table of contents
 \newpage
 \tableofcontents

%%% start a new page and display the list of figures
 \newpage
 \listoffigures

%%% start a new page and display the list of tables
 %\newpage
 %\listoftables

%%% display the main document on a new page
 \newpage

 \pagenumbering{arabic} %% normal page numbers (include it, if roman was used above)

%%%%%%%%%%%%%%%%%%%%%%%%%%%%%%%%%%%%%%%%%%%%%%%%%%%%%%%%%%%%%%%%%%%%%%%%%%%%%%%%
%%%
%%% begin main document
%%% structure: \section \subsection \subsubsection \paragraph \subparagraph
%%%

%\section*{Zusammenfassung}
%Nico
\section{Aufgabenstellung}
Bei dem \glqq Cat-Mouse-Cheese\grqq{} Projekt gibt es zwei Roboter: eine Katze und eine Maus. Desweiteren befindet sich ein Stück Käse auf dem Spielfeld.
Ziel der Katze ist es die Maus zu fangen und zu verhindern, dass die Maus den Käse bekommt.
Die Maus muss dementsprechend der Katze ausweichen und zum Käse kommen.
Das Spiel gilt als beendet, sobald:
\begin{description}
\item[a)]Die Katze die Maus gefangen hat => Die Katze siegt
\item[b)] Die Maus kommt zum Käse => Die Maus siegt
\item[c)] Nach einer gewissen Zeit tritt weder a) noch b) ein => Unentschieden
\end{description}
In diesem Projekt wurde die Rolle der Katze übernommen.

\subsection{Anforderungen}
Bei den verwendeten Robotern handelt es sich jeweils um einen Turtlebot3 Burger.\\ 
Dieser Roboter hat eine maximale Lineargeschwindigkeit von 0.22 m/s und eine maximale Winkelgeschwindigkeit von 2.84 rad/s. \footnote{\url{https://emanual.robotis.com/docs/en/platform/turtlebot3/specifications/\#hardware-specifications}}\\
Die Drehung des Roboters wird durch einen Radianten \footnote {Wertebereich [0:2 $\pi$]} angegeben. Ein positiver Wert führt zu einer Drehung gegen den Uhrzeigersinn, dementsprechend führt ein negativer Wert zu einer Drehung in Uhrzeigersinn.
Um Hindernisse zu erkennen, besitzt er einen Laserscanner mit einer Scanreichweite von 120 bis 3500 mm und einer Auflösung von 1 Grad. \footnote {1 Gard ~ 0,0174533} \footnote{\url{https://emanual.robotis.com/docs/en/platform/turtlebot3/appendix_lds_01/}} 
\\\\
Für das Spiel gibt es jedoch Restriktionen der Roboter bezüglich Linear- und Winkelgeschwindigkeit.\\
\paragraph{Katze:}
\begin{description}
 \item Lineargeschwindigkeit: konstant 0.22 m/s
 \item Winkelgeschwindigkeit: -0.8 ... 0.8 rad/s
\end{description}    
\paragraph{Maus:}
\begin{description}
 \item Lineargeschwindigkeit: konstant 0.18 m/s
 \item Winkelgeschwindigkeit: -2.84 .. 2.84 rad/s
\end{description}
\clearpage
\subsubsection{Karten}
Für das Projekt wurden seches Karten bereitgestellt. Drei für das Spiel und drei weitere Karten um Grenzfälle zu testen.  Die drei Grenzfallkarten wurden bei der Entwicklung nicht berücksichtigt daher werden sie im folgenden nicht behandelt.
\mygraphics{0.5\textwidth}{Welt1.png}{Welt 1}
\noindent Die erste Welt (\ref{fig:Welt1.png}) ist der stilistischen Sicht einer Schildkröte nachempfunden. Die Welt ist durch eine sechsseitige Mauer eingegrenzt. Jede Ecke ist durch eine Säule abgerundet. In der Mitte der Welt befinden sich neun Säulen, die in einem Raster angeordnet sind. Der Käse ist an der gegenüberliegenden Seite zur Maus positioniert. Die Katze befindet nahe dem Käse leicht seitlich zwischen Maus und Käse.\\ \clearpage
\mygraphics{0.5\textwidth}{Welt2.png}{Welt 2}
\noindent Die zweite Welt (\ref{fig:Welt2.png}) ist quadratisch eingemauert und besitzt vier Säulen in der Kartenmitte. Katze und Maus sind an gegenüberliegenden Ecken positioniert. Der Käse liegt in der Nähe der Maus, diese blickt jedoch in die entgegengesetzte Richtung.\\
\mygraphics{0.5\textwidth}{Welt3.png}{Welt 3}
\noindent In der dritten Welt (\ref{fig:Welt3.png}) befinden sich nur die Katze, die Maus und der Käse.
Alle drei befinden sich auf einer Geraden.\clearpage
\subsection{Mögliche Implementierungsmöglichkeiten}
Im Laufe des Semester wurden verschiedene Methoden, das Verhalten des Roboters zu implementieren, in der Vorlesung vorgestellt.
\subsubsection{Kraftbasierte Kollisionsvermeidung und Homing}
Bei diesen Verfahren wird bei der Kollisionsvermeidung eine Kraft berechnet, die von den Hindernissen auf den Roboter wirken. Je näher der Roboter an einem Hindernis ist, desto mehr wird er von diesem  abgestoßen. Beim Homing erfährt der Roboter eine Anziehungskraft auf einen Punkt im Spielfeld. Durch das Aufaddieren beider Kräfte, dem Homing und der Kollisionsvermeidung, ergibt sich die endgültige Kraft, die den Roboter von Hindernissen abstößt ihn jedoch gleichzeitig immer näher zum Ziel treibt.
\subsubsection{Sense Plan Act}
Hier wird für jede Folge von möglichen Zuständen der maximale Nutzen berechnet, der Roboter arbeitet somit vorausschauend. Dabei entsteht ein Suchbaum. Dieser Suchbaum kann mit dem MinMax-Algorithmus optimiert werden.
\subsubsection{Optimal Control}
Während bei MinMax die addierten Kosten minimiert oder der addierte Reward maximiert wird, werden bei optimal control die integrierten Kosten minimiert.
\subsubsection{Path Planning}
Durch eine gegebene Umgebungskarte versucht der Roboter den kürzesten Weg zu seinem Ziel zu berechnen. Um dies zu erreichen wird die Karte beispielsweise mit einem Occupancy grid in Zellen zerlegt und mit einem Shortest Path Algorithmus wie dem A*-Algorithmus die kürzeste Distanz vom Start zum Ziel berechnet.  
\subsection{Zielsetzung}
Im Rahmen dieses Projekts wird ein kraftbasierter Ansatz implementiert. Im Laufe des Projekts wurden die Parameter so angepasst, dass sie möglichst optimal im Wettkampf gegen die Ansätze der anderen Teilnehmer des Projektes abschneiden.
\clearpage
\section{Konzept}
\subsection{Grundverhalten}
Die Katze hat zwei wesentliche Grundbestandteile. Sie kann auf einen gegebenen Punkt  \glqq homen\grqq{} und sie besitzt eine Kollisionsvermeidung.
\subsection{Homing}
\subsubsection{Ziele}
F"ur das Homing gibt es vier verschiedene Ziele (\ref{fig:homingZiele.png}). 
\begin{enumerate}
\item der K"ase
\item der Mittelpunkt zwischen K"ase und Maus
\item der vermutete Punkt auf den sich die Maus zubewegt
\item die Maus
\end{enumerate}
\mygraphics{0.7\textwidth}{homingZiele.png}{Ziele f"ur das Homing}
\clearpage
\subsubsection{Zustandsautomat}
Die Zielauswahl wird durch einen Zustandsautomaten (\ref{fig:stateMachine.png}) festgelegt.
Der Zustandsautomat besitzt drei Zust"ande, mit dem Zustand \glqq go to cheese\grqq{} als Startzustand. Im Zustand \glqq go to cheese\grqq{} ist der K"ase das Ziel, im Zustand \glqq go to mid\grqq{} ist der Mittelpunkt zwischen Maus und K"ase das Ziel und der Zustand \glqq hunt\grqq{} fasst die zwei Ziele Maus und den vermutete Punkt auf den sich die Maus zubewegt zusammen. Wie der Zustand \glqq hunt\grqq{} sich verh"alt wird im Abschnitt \ref{hunt} genauer erl"autert.\\
\hspace*{0.63\textwidth}\scalebox{0.7}{\begin{tabular}[h]{|c|}
\hline
m\_c\_d := Maus-Katze Distanz \\ 
m\_ch\_d := Maus-Käse Distanz \\ 
c\_ch\_d := Katze-Käse Distanz \\
\hline
\end{tabular}}
\mygraphics{0.7\textwidth}{stateMachine.png}{Zustandsautomat}
\noindent Es gibt drei Kriterien die einen Zustandswechsel zur Folge haben. Das \glqq arrived\grqq{} Kriterium ist hierbei das einfachste und wird erreicht sobald ein gewisser Abstand zum Zielpunkt unterschritten wird.\\
Das zweite Kriterium \glqq$m\_c\_d < P_{1} \land  m\_ch\_d > P_{2} $\grqq{} besagt: sollte der Abstand zwischen Katze und Maus einen gewissen Wert $P_{1}$ unterschritten haben und zus"atzlich der Abstand zwischen K"ase und Maus einen weiteren Wert $P_{2}$ "uberschritten haben so wird der Zustand gewechselt. Umgangssprachlich ausgedr"uckt bedeutet es, sollte sich die Katze nah an der Maus befinden und die Maus aber noch weit genug vom K"ase weg sein so wird die Maus gejagt. Wobei der Parameter $P_{1}$ \glqq nah \grqq{} und der Parameter $P_{2}$ \glqq noch weit genug\grqq{} definiert. 
$P_{1}$ und $P_{2}$ wurden empirisch festgelegt.\\
Das dritte Kriterum \glqq $m\_c\_d + P_{3} < c\_ch\_d$\grqq{} bedeutet: ist der Abstand zwischen der Maus und dem Käse kleiner dem Abstand zwischen der Katze und dem Käse so erfolgt der Zustandswechsel. Hier wird ein Wert $P_{3}$ hinzugefügt der als Hysterese fungiert.\\
Die Parameter $P_{1}$, $P_{2}$ und $P_{3}$ wurden empirisch festgelegt.\\
\\\clearpage
\noindent Im Zustand \glqq go to cheese\grqq{} löst das \glqq arrived\grqq{} Kriterium einen Wechsel in den Zustand \glqq go to mid\grqq{} aus. Äquivalent dazu löst das gleiche Kriterium im Zustand \glqq go to mid\grqq{} den Wechsel in den Zustand \glqq hunt\grqq{} aus.
Von beiden Zuständen, \glqq go to cheese\grqq{} und \glqq go to mid\grqq{}, findet ein Wechsel in den Zustand \glqq hunt\grqq{} statt, sollte das zweite Kriterium, \glqq$m\_c\_d < P_{1} \land  m\_ch\_d > P_{2} $\grqq{}, erfüllt sein. Im Zustand \glqq hunt\grqq{} greift das dritte Kriterium, \glqq $m\_c\_d + P_{3} < c\_ch\_d$\grqq{}, um den Wechsel in den Zustand \glqq go to cheese\grqq{} einzuleiten.
\subsubsection{Zustand: \glqq hunt\grqq{}}
\label{hunt}
Im Zustand \glqq hunt\grqq{} wird entschieden ob die Maus direkt als Ziel angefahren wird oder ob die Katze die Bewegung der Maus vorausberechnet. Im Fall der Vorausberechnung wird auch entschieden wie weit vorausberechnet wird.\\
Die Berechnung der Bewegung bezweckt eine sich vom Käse entfernende Schlingerbewegung der Maus (\ref{fig:ziel.png}).
Dieser Ansatz fungiert als Gegenspieler zu einem kraftbasierten Ansatz der Maus.
\mygraphics{0.6\textwidth}{ziel.png}{Schlingerbewegung}
\noindent Um diese Verhalten zu erreichen wird die Bewegung der Maus verschieden weit vorausberechnet.
Im Zustand  hunt \grqq{} wird, bis zu einem gewissen Schwellwert, nicht direkt die Maus angefahren sondern ein vorausberechneter Punkt vor der Maus. Je näher die Katze der Maus kommt desto weiter steuert sie vor die Maus. Unterschreitet die Distanz zwischen Katze und Maus einen gewissen Schwellwert, so wird direkt die Maus als Ziel gesetzt.
\clearpage
\noindent Den Wert der Skalierung der Vorausberechnung wird r-Wert genannt. Er wird anhand einer Funktion berechnet.
Gegeben aus den Anforderungen benötigt besagte Funktion eine sägezahnähnliche Form.\\
Hierzu wurde die Funktion $r= \frac{2}{x} - \frac{2}{x^3}$
gewählt mit $x := Distanz(Maus,Katze)$  (\ref{fig:huntFunction.png}).
\mygraphics{0.9\textwidth}{huntFunction.png}{r-Faktor Funktion}
\clearpage
\subsection{Collison Avoidance}
Bei der Kollisionsvermeidung werden nicht alle 360 Sensorwerte berücksichtigt. Der relevante Bereich wird durch einen Parameter \footnote {rel\_phi} angegeben. Objekte die im Scanbereich sind haben eine abstoßende Wirkung auf die Katze.
\subsubsection{Funktionsentwicklung}
\label{collision}
Folgende Kriterien sind für die Funktion der Kollisionsvermeidung wichtig.
\begin{itemize}
\item Je geringer die Distanz zu einem Objekt \footnote {sensor.range} desto größe die Kraft
\item Je geringer der Winkel eines Objekts zur Katze, desto größer deren Kraft \footnote{Ein Objekt in Fahrtrichtung soll eine größere Kraft haben als ein Objekt welches sich Seitlich befindet}
\end{itemize}
Nun ist ein genauer Blick auf die Wertebereiche der relevanten Variablen nötig.\\
\begin{tabular}[h]{|l|c|r|}
\hline
Variable & Wert & Auswirkung auf die Kraft \\
sensor.angular & $[0 : rel\_phi]$ & negativ (Drehung mit den Uhrzeigersinn) \\
sensor.angular & $[2\pi : 2\pi-rel\_phi]$ & positiv (Drehung gegen den Uhrzeigersinn) \\
sensor.range & gegen 0,8  & Kraft gegen 0 \\
sensor.range & gegen 0 & Kraft wird größer \\
\hline
\end{tabular}
\\
\\
Folgende Funktion erfüllt die aufgestellten Kriterien und weist die gewünschten Verhaltensweise auf.
\[ Force= -\sin(sensor.angular) * (rel\_range-sensor.range) \]
\mygraphics{0.7\textwidth}{Kollision.png}{Funktion
Collison Avoidance}
\noindent Die negative Sinusfunktion wird hier als Grundgerüst genommen, sa sich sensor.angle zwischen 0 und $2\pi$ bewegt. Werte nahe der 0 führen zu einer negativen Kraft und bewirken somit eine Rechtsdrehung. Werte nahe $2\pi $ bewirken hingegen eine Linksdrehung. Die so entstehende winkelabhängige Kraft wird  nun mit der Distanz multipliziert, von der vorher von der Maximaldistanz abgezogen wird. Somit ist der Multiplikator größer, je kleiner die Distanz ist.
\subsubsection{Sonderfälle}
Die Maus wird von den Sensoren erfasst und muss daher aus der Kollisionsvermeidung herausgerechnet werden.\\
Um dieses Problem zu lösen wurde ein Spezialfall entwickelt.
Ziel ist es alle Sensorwerte, die die Maus erfassen zu bestimmen und diese in der Kollisionsvermeidung nicht  zu berücksichtigen. \\
\paragraph{Strategie:}
\begin{enumerate}
\item Maus ist in Scanreichweite
\item Gescannte Range entspricht der Entfernung zur Maus
\item Relevante Winkle in Bezug auf die Größe der Maus berechnen
\item Alle sensor.range-Werte der relevanten Winkel auf Infinity setzen
\end{enumerate}
Sobald die Maus in Sensorreichweite ist, ist es wichtig zu überprüfen ob die Sensor.range der mathematischen Distanz zur Maus entspricht. Falls man dies nicht überprüft kann es passieren, dass sich ein Hindernis zwischen Maus und Katze befinden. Das Hindernis wird von dem Algorithmus als Maus tituliert und für die Kollisionsvermeidung nicht berücksichtigt. Dies hat zur Folge, dass die Katze in das Hindernis hineinfährt.\\
Sofern diese Fälle berücksichtigt worden sind, kann es daran gehen die relevanten Sensorwerten zu löschen, dies bedeutet sie auf Infinity zu setzen. \\
Zuerst wird der angle\_mouse berechnet dies ist der Winkel aus Sicht der Katze zur Maus, mit den globalen Positionsangabne\footnote{Positionen der Maus in odom Sysem}.
\[ angle\_mouse=\tan( \frac{\Delta x}{\Delta y} ) -cat\_phi \]
Angle\_mouse gibt den Winkel der Maus zur Basis der Katze an, nun wird berechnet welcher Sensorwinkel auf die Mitte der Maus zeigt:
\[ laserMouse\_middle  =  \vert \frac{ (angle\_mouse + 2  \pi) \% (2 \pi)}{angle\_increment}  \vert * angle\_increment \]
\mygraphics{0.5\textwidth}{collisionANGLE2.png}{Sensorwinkel Katze}
\noindent Nachdem der Sensorwinkel zu Mitte der Maus berechnet wurde, wird mit Hilfe des Radiuses(mouse\_radius) der Maus die Außenwinkel berechnet.
\[
laserMous\_border =  \vert { \tan( \frac{mouse\_radius}{\Delta cat\_mouse} ) } / angle\_increment \vert  * angle\_increment \]
\hspace*{0.63\textwidth}\scalebox{0.7}{\begin{tabular}[h]{|c|}
\hline
Winkel zwischen Rot \& Lila:= $\tan( \frac{mouse\_radius}{\Delta cat\_mouse} )$ \\
Blau := mouse\_radiu s\\
Rot := $\Delta cat\_mouse$ \\
\hline
\end{tabular}}
\mygraphics{0.5\textwidth}{catmous.png}{Beziehungen Katze zu Maus}
\noindent Nachdem berechnet wurde welche sensor.angle-Werte momentan die Maus scannen, werden alle Sensor.range-Werte dieser Winkel auf Infinity gesetzt. Nachdem alle relevanten Werte des     Sensors auf Infinity gesetzt sind, ist die Maus für die Kollisionsvermeidung unsichtbar und die Katze sieht die Maus nicht als Hindernis an, welche ihre Fahrtrichtung beeinflusst.\\
\mygraphics{0.5\textwidth}{collisionMOUS.png}{Lasersensor ignorieren }
\noindent Ein weiterer Spezialfall ist der Käse. Er ist zu niedrig um vom Lasersensor erfasst zu werden.
\paragraph{Strategie:}
\begin{enumerate}
\item Käse ist in Scanreichweite
\item Relevante Winkle in Bezug auf die Größe und Position des Käses berechnen
\item Alle sensor.range-Werte der relevanten Winkel Werte zuweisen
\end{enumerate}
Mit dem Globalen Koordinaten wird überprüft ob der Käse theoretisch in Sensorreichweite ist. Parallel zum Vorgehen bei der Maus wird der Winkel vom Käse zur Katze und der Sensorwinkle zur Mitte des Käses berechnet.
\[ angle\_cheese=\tan( \frac{\Delta x}{\Delta y} ) -cat\_phi \]
\[ laserMouse\_cheese  =  \vert \frac{ (angle\_cheese + 2  \pi) \% (2 \pi)}{angle\_increment}  \vert * angle\_increment \]
Nachdem berechnet wurde wo der Käse sich befindet, werden mithilfe der Kreisfunktion die sensor.range-Werte berechnet und gesetzt. $ sensor.range=\sqrt[]{ rad^2 -x^2 }$
\[ sensor.range_i = \Delta cheese - \sqrt[]{ cheese\_rad^2 -\frac{\tan( laser\_cheesMiddl - angle_i) }{\Delta cheese}^2 }\]
\hspace*{0.63\textwidth}\scalebox{0.7}{\begin{tabular}[h]{|c|}
\hline
Blau := $\frac{\tan( laser\_cheesMiddl - angle_i) }{\Delta cheese}$ \\
Rot := $\Delta cat\_cheese$ \\
Lilia := $sensor.range_i$ \\
\hline
\end{tabular}}
\mygraphics{0.5\textwidth}{catcheese.png}{Überblick Katze Käse}
\noindent Nachdem alle Werte berechnet sind kann die normale Kollisionsvermeidung durchgeführt werden. Die Katze registriert nun dort wo sich der Käse befindet einen Halbkreis und weicht diesem Objekt aus. 
Es wird der Katze eine Säule an der Position des Käse vorgetäuscht.
\mygraphics{0.5\textwidth}{collisionCHEESE.png}{Käse als Hindernis Simuliert}\clearpage
\section{Diskussion und Ausblick}
\subsection{Bewertung}
Der kraftbasierte Ansatz erweist sich nach der Auswertung auf den Welten eins bis drei als praktikabel. Durch seinen rein reaktiven Charakter ist er weniger rechenintensiv als pfadplanende Ansätze. 
Der Erfolg ist allerdings sehr abhängig von den Karten und den implementierten Verhalten der Gegenspieler. Für komplexerer Karten, wie Szenario A-C, wäre ein Planungsalgorithmus vorteilhafter. Bei der Sichtung der Videos ist desweiteren aufgefallen, dass keine der Gegenspieler ihre volle Drehgeschwindigkeit nutzen und dadurch möglicherweise Potenzial verschenken
\subsubsection{Fehlerhafter Zustandsautomat}
Bei der Auswertung des Videos c\_1\_15\_1\_c  ist zu sehen, dass der Zustandsautomat nicht differenziert genug ist. In dem Video sieht man, wie die Katze aus dem Zustand \glqq hunt\grqq{} in den Zustand \glqq go to cheese\grqq{} wechselt, da die Maus näher am Käse ist als die Katze. Nachdem die Katze nun zum Käse fährt und diesen erreicht wechselt die Katze wieder in den Zustand \glqq hunt \grqq{}. Zu diesem Zeitpunkt ist die Maus jedoch schon zu nahe am Käse und die Drehgeschwindigkeit der Katze reicht nicht aus um die Maus noch zu erreichen.
Das Problem ließe sich auf verschiedene Weise lösen, zeigt jedoch das grundsätzliche Problem, dass der Zustandsautomat nicht genügend Grenzfälle abfängt. Ein Ansatz wäre, vor dem Wechsel  vom Zustand \glqq hunt \grqq{} in den Zustand \glqq go to cheese\grqq{} den Abstand von Maus zur Katze mit der Distanz von Maus zum Käse zu vergleichen.
\subsubsection{Recovery Strategie}
Bei Karten mit komplexen, polygonen Hindernissen oder Sackgassen (siehe Szenario A) ist die Katze zum jetzigen Zeitpunkt nicht in der Lage darauf zu reagieren. Eine Recovery Strategie für Sackgassen fehlt komplett. Eine Strategie ließe sich mittels neuer Zustände oder äquivalent mit einer Verhaltensfusion integrieren. Für komplexe, polygone Hindernisse wäre ein Path Planning Ansatz vorteilhafter.
\subsubsection{Sinusfunktion}
Bei näherer Betrachtung der Funktion zur Berechnung der Kraft (\ref{collision}) bei der Kollisionsvermeidung fällt auf, dass bei Winkelwerten gegen 0 rad und gegen $\pi /2$ rad die Funktion gegen 0 Rad geht. Dies hat zur Folge, dass Objekte nahe der Fahrtrichtung wenig bis keine Abstoßungskraft haben.  Objekte die bei einem Winkel von $ \frac{2}{\pi}$ rad und $3 \frac{\pi}{2} $ rad  liegen haben dagegen die Maximale Abstoßungskraft(Extrempunkte).
Lösen ließe sich dieses Problem durch die Streckung der Funktion in x-Richtung, damit die Extrempunkte bei 0 und $ \frac{2}{\pi}$ liegen.
\subsection{Ausblick und Verbesserungen}
Beim kraftbasierten Ansatz ergeben sich sehr viele Parameter. Die Optimierung dieser ist von elementarer Bedeutung für den Erfolg des Ansatzes.
Die Parameter könnten mit Hilfe eines dafür konstruierten Neuronalen Netzes effektiver optimiert werden. Hierzu müsste jedoch eine eigene Simulation implementiert werden, da Gazebo hierfür zu langsam ist.\\
Des Weiteren könnte bei der Implementierung  zur Vereinfachung der Zustandsautomat in eine Verhaltensfusion übertragen werden. Somit könnten die einzelnen Verhalten als eigenständige Ros nodes implementiert werden. Dadurch könnte man die Verhaltensweisen einzeln optimieren und situationsbedingt gewichten.
%%%
%%% end main document
%%%
%%%%%%%%%%%%%%%%%%%%%%%%%%%%%%%%%%%%%%%%%%%%%%%%%%%%%%%%%%%%%%%%%%%%%%%%%%%%%%%%
% \appendix  %% include it, if something (bibliography, index, ...) follows below
%%%%%%%%%%%%%%%%%%%%%%%%%%%%%%%%%%%%%%%%%%%%%%%%%%%%%%%%%%%%%%%%%%%%%%%%%%%%%%%%
%%%
%%% bibliography
%%%
%%% available styles: abbrv, acm, alpha, apalike, ieeetr, plain, siam, unsrt
%%%
% \bibliographystyle{plain}
%%% name of the bibliography file without .bib
%%% e.g.: literatur.bib -> \bibliography{literatur}
% \bibliography{FIXXME}
\end{document}
%%% }}}
%%% END OF FILE
%%%%%%%%%%%%%%%%%%%%%%%%%%%%%%%%%%%%%%%%%%%%%%%%%%%%%%%%%%%%%%%%%%%%%%%%%%%%%%%%
%%% Notice!
%%% This file uses the outline-mode of emacs and the foldmethod of Vim.
%%% Press 'zi' to unfold the file in Vim.
%%% See ':help folding' for more information.
%%%%%%%%%%%%%%%%%%%%%%%%%%%%%%%%%%%%%%%%%%%%%%%%%%%%%%%%%%%%%%%%%%%%%%%%%%%%%%%%
%% Local Variables:
%% mode: outline-minor
%% OPToutline-regexp: "%% .*"
%% OPTeval: (hide-body)
%% emerge-set-combine-versions-template: "%a\n%b\n"
%% End:
%% vim:foldmethod=marker
